% \iffalse meta-comment
% !TeX program  = XeLaTeX
% !TeX encoding = UTF-8
%<*internal>
\iffalse
%</internal>
%<*readme>
apfontspec
==========

The `apfontspec` package provides an similar interface to `fontspec`
package for loading CJK fonts in [ApLaTeX](https://github.com/clerkma/ptex-ng).

Basic Usage
-----------
The package provides the following commands to define a CJK font family:

    \setCJKmainfont{<font name>}[<font options>]
    \setCJKsansfont{<font name>}[<font options>]
    \setCJKmonofont{<font name>}[<font options>]
    \setCJKfamilyfont{<font family>}{<font name>}[<font options>]
    \newCJKfontfamily\<font switch>{<font name>}[<font options>]
    \CJKfontspec{<font name>}[<font options>]
    \defaultCJKfontfeatures{<default font options>}
    \addCJKfontfeatures{<font options to add>}

You can read the package manual (in Chinese) for more detailed explanations.

Contributing
------------

[Issues](https://github.com/CTeX-org/apfontspec/issues) and
[pull requests](https://github.com/CTeX-org/apfontspec/pulls)
are always welcome.

Copyright and Licence
---------------------

    Copyright (C) 2020 by Qing Lee <sobenlee@gmail.com>
    -----------------------------------------------------------------

    This work may be distributed and/or modified under the
    conditions of the LaTeX Project Public License, either
    version 1.3c of this license or (at your option) any later
    version. This version of this license is in
       http://www.latex-project.org/lppl/lppl-1-3c.txt
    and the latest version of this license is in
       http://www.latex-project.org/lppl.txt
    and version 1.3 or later is part of all distributions of
    LaTeX version 2005/12/01 or later.

    This work has the LPPL maintenance status "maintained".

    The Current Maintainer of this work is Qing Lee.

    This package consists of the file  apfontspec.dtx,
                 and the derived files apfontspec.pdf,
                                       apfontspec.sty,
                                       apfontspec.ins and
                                       README.md (this file).
%</readme>
%<*internal>
\fi
\begingroup
  \def\temp{LaTeX2e}
\expandafter\endgroup\ifx\temp\fmtname\else
\csname fi\endcsname
%</internal>
%<*install>

\input ctxdocstrip %

\preamble

    Copyright (C) 2020 by Qing Lee <sobenlee@gmail.com>
-----------------------------------------------------------------

    This work may be distributed and/or modified under the
    conditions of the LaTeX Project Public License, either
    version 1.3c of this license or (at your option) any later
    version. This version of this license is in
       http://www.latex-project.org/lppl/lppl-1-3c.txt
    and the latest version of this license is in
       http://www.latex-project.org/lppl.txt
    and version 1.3 or later is part of all distributions of
    LaTeX version 2005/12/01 or later.

    This work has the LPPL maintenance status "maintained".

    The Current Maintainer of this work is Qing Lee.

-----------------------------------------------------------------

\endpreamble

\postamble

    This work consists of the file  apfontspec.dtx,
              and the derived files apfontspec.pdf,
                                    apfontspec.sty,
                                    apfontspec.ins and
                                    README.md.
\endpostamble

\generate
  {
%</install>
%<*internal>
    \usedir{source/uplatex/apfontspec}
    \file{apfontspec.ins} {\from{\jobname.dtx}{install}}
%</internal>
%<*install>
    \usedir{tex/uplatex/apfontspec}
    \file{apfontspec.sty} {\from{\jobname.dtx}{package}}
    \nopreamble\nopostamble
    \usedir{doc/uplatex/apfontspec}
    \file{README.md}      {\from{\jobname.dtx}{readme}}
  }

\endbatchfile
%</install>
%<*internal>
\fi
%</internal>
%<package>\NeedsTeXFormat{LaTeX2e}
%<package>\RequirePackage{expl3}
%<+package>\GetIdInfo$Id$
%<package>  {CJK Font Selection for ApLaTeX}
%<package>\ProvidesExplPackage{\ExplFileName}
%<package>  {\ExplFileDate}{0}{\ExplFileDescription}
%<*driver>
\documentclass{ctxdoc}
\ExplSyntaxOn
\DeclareDocumentCommand \gitsha { m }
  {
    \href { https \c_colon_str //github.com/CTeX-org/apfontspec/#1 }
          { rev. ~ \texttt{#1} }
  }
\ExplSyntaxOff
\begin{document}
  \DocInput{\jobname.dtx}
  \IndexLayout
  \PrintIndex
\end{document}
%</driver>
% \fi
%
% \CheckSum{416}
% \GetFileId{apfontspec.sty}
%
% \title{\bfseries\pkg{apfontspec} 宏包}
% \author{李清}
% \date{\ExplFileDate\thanks{\gitsha{\ExplFileVersion}.}}
% \maketitle
%
%
% \begin{abstract}
%
% \pkg{apfontspec} 宏包提供了 \ApLaTeX 格式下的比较方便的 CJK 字体设置界面。
% 界面的语法形式与 \pkg{fontspec} 宏包类似。
%
% \end{abstract}
%
% \begin{documentation}
%
% \end{documentation}
%
%
% \StopEventually{}
%
%
%\begin{implementation}
%
% \section{代码实现}
%
%    \begin{macrocode}
%<*package>
%    \end{macrocode}
%
%    \begin{macrocode}
%<@@=apfs>
%    \end{macrocode}
%
%    \begin{macrocode}
\prop_gput:Nnn \g_msg_module_name_prop { apfs } { apfontspec }
%    \end{macrocode}
%
% 检查是否为 \ApLaTeX。
%    \begin{macrocode}
\msg_new:nnn { apfs } { aptex-required }
  {
    The~apfontspec~package~requires~engine~ApTeX.\\\\
    You~must~change~your~typesetting~engine~to~`aplatex'.\\
    Loading~apfontspec~will~abort!
  }
\cs_if_exist:NF \ngbanner
  { \msg_critical:nn { apfs } { aptex-required } }
%    \end{macrocode}
%
%    \begin{macrocode}
\RequirePackage { xparse }
%    \end{macrocode}
%
% \begin{variable}[int]{\l_@@_tmp_tl}
% 临时变量。
%    \begin{macrocode}
\tl_new:N \l_@@_tmp_tl
%    \end{macrocode}
% \end{variable}
%
% \begin{macro}{\apfs_nfss_declare_family:nnn,
% \apfs_nfss_declare_shape:nnnnnn}
% \pkg{NFSS} 字体命令的宏包内部函数。
%    \begin{macrocode}
\cs_new_protected:Npn \apfs_nfss_declare_family:nnn
  { \DeclareKanjiFamily }
\cs_new_protected:Npn \apfs_nfss_declare_shape:nnnnnn
  { \DeclareFontShape@ }
%    \end{macrocode}
% \end{macro}
%
% \begin{macro}{\apfs_define:n, \apfs_set:n,
%  \apfs_set_filter:nn, \apfs_set_features:n,
%  \apfs_set_size_features:n, \apfs_set_groups:nn}
%    \begin{macrocode}
\cs_new_protected:Npn \apfs_define:n
  { \keys_define:nn { apfs } }
\cs_new_protected:Npn \apfs_set:n
  { \keys_set:nn { apfs } }
\cs_new_protected:Npn \apfs_set_filter:nn
  { \keys_set_filter:nnn { apfs } }
\cs_new_protected:Npn \apfs_set_features:n
  { \keys_set_filter:nnn { apfs } { main , fontname } }
\cs_new_protected:Npn \apfs_set_size_features:n
  { \keys_set_filter:nnn { apfs } { main , size } }
\cs_new_protected:Npn \apfs_set_groups:nn
  { \keys_set_groups:nnn { apfs } }
%    \end{macrocode}
% \end{macro}
%
% \begin{variable}[int]{\c_@@_prefix_tl}
% 保存字体前缀,固定为 |ot:|。
%    \begin{macrocode}
\tl_const:Nn \c_@@_prefix_tl { ot: }
%    \end{macrocode}
% \end{variable}
%
% \begin{variable}[int]{\l_@@_yoko_encoding_tl,\l_@@_tate_encoding_tl}
% 保存横排与直排字体的 \pkg{NFSS} 编码名字,初始化为 |JY2| 和 |JT2|。
%    \begin{macrocode}
\tl_new:N \l_@@_yoko_encoding_tl
\tl_new:N \l_@@_tate_encoding_tl
\tl_set:Nn \l_@@_yoko_encoding_tl { JY2 }
\tl_set:Nn \l_@@_tate_encoding_tl { JT2 }
%    \end{macrocode}
% \end{variable}
%
% \begin{variable}[int]{\l_@@_yoko_font_tl,\l_@@_tate_font_tl}
% 保存横排与直排字体的文件名。
%    \begin{macrocode}
\tl_new:N \l_@@_yoko_font_tl
\tl_new:N \l_@@_tate_font_tl
%    \end{macrocode}
% \end{variable}
%
% \begin{macro}{NFSSEncoding}
% 设置字体的 \pkg{NFSS} 编码名字。
%    \begin{macrocode}
\apfs_define:n
  {
    NFSSEncoding   .code:n = \@@_assign:nn { encoding } {#1} ,
    NFSSEncoding .groups:n = main
  }
%    \end{macrocode}
% \end{macro}
%
% \begin{macro}{Font}
% 用于实现 \opt{FontFace} 和 \opt{SizeFeatures} 选项中的 \opt{Font}。
%    \begin{macrocode}
\apfs_define:n
  {
    Font .tl_set:N = \l_@@_fontname_tl ,
    Font .groups:n = fontname
  }
\cs_new_protected:Npn \@@_extract_fontname:n
  {
    \tl_clear:N \l_@@_fontname_tl
    \apfs_set_groups:nn { fontname }
  }
%    \end{macrocode}
% \end{macro}
%
% \begin{macro}{\apfs_new_feature:nn}
% 定义字体选项,有横排和直排字体两种形式。|#1| 是选项名称,|#2| 用于命名内部 tl 变量。
%    \begin{macrocode}
\cs_new_protected:Npn \apfs_new_feature:nn #1#2
  {
    \tl_new:c { l_@@_yoko_ #2 _tl }
    \tl_new:c { l_@@_tate_ #2 _tl }
    \apfs_define_feature:nn {#1} {#2}
  }
\cs_new_protected:Npn \apfs_define_feature:nn #1#2
  {
    \apfs_define:n
      {
        #1 .code:n   = \@@_assign:nn {#2} { ##1 } ,
        #1 .groups:n = feature
      }
  }
\cs_new_protected:Npn \apfs_new_feature_code:nnn #1#2#3
  {
    \tl_new:c { l_@@_yoko_ #2 _tl }
    \tl_new:c { l_@@_tate_ #2 _tl }
    \apfs_define:n
      {
        #1 .code:n   = #3 ,
        #1 .groups:n = feature
      }
  }
%    \end{macrocode}
% \end{macro}
%
% \begin{macro}{Index,RawFeature,Scale,JFM,Extension,LoadingSetting}
% 设置 \file{.ttc} 字体的索引。
%    \begin{macrocode}
\apfs_new_feature:nn { Index }          { index }
\apfs_new_feature:nn { RawFeature }     { feature }
\apfs_new_feature:nn { Scale }          { scale }
\apfs_new_feature:nn { JFM }            { jfm }
\apfs_new_feature:nn { Extension }      { ext }
\apfs_new_feature:nn { LoadingSetting } { loading }
%    \end{macrocode}
% \end{macro}
%
% \begin{macro}{Script,Language}
% \opt{Script} 和 \opt{Language} 可接受 Chinese Simplified 等自然名字。
%    \begin{macrocode}
\apfs_new_feature_code:nnn { Script } { script }
  { \@@_extract_alias:nn { script } {#1} }
\apfs_new_feature_code:nnn { Language } { language }
  { \@@_extract_alias:nn { language } {#1} }
%    \end{macrocode}
% \end{macro}
%
% \begin{macro}{SizeFeatures}
% \opt{SizeFeatures} 选项的格式与 \pkg{fontspec} 略有差别。
% \begin{syntax}
%   SizeFeatures =
%  \  \{
%  \    ...
%  \    \Arg{size range} \Arg{features} ,
%  \    \Arg{size range} \{ Font = \meta{file name} , \meta{features} \} ,
%  \    ...
%  \  \}
% \end{syntax}
%    \begin{macrocode}
\apfs_define:n
  {
    SizeFeatures .clist_set:N = \l_@@_size_features_clist ,
    SizeFeatures    .groups:n = { feature , size }
  }
%    \end{macrocode}
% \end{macro}
%
% 定义主要选项。
%    \begin{macrocode}
\apfs_define:n
  {
    FontFace            .code:n = \@@_save_fontface:nnn #1 ,
    UprightFont         .code:n = \@@_save_shape:nnn { md } { shape } {#1} ,
    ItalicFont          .code:n = \@@_save_shape:nnn { md } { it } {#1} ,
    SlantedFont         .code:n = \@@_save_shape:nnn { md } { sl } {#1} ,
    BoldFont            .code:n = \@@_save_shape:nnn { bf } { shape } {#1} ,
    BoldItalicFont      .code:n = \@@_save_shape:nnn { bf } { it } {#1} ,
    BoldSlantedFont     .code:n = \@@_save_shape:nnn { bf } { sl } {#1} ,
    UprightFeatures     .code:n = \@@_save_features:nnn { md } { shape } {#1} ,
    ItalicFeatures      .code:n = \@@_save_features:nnn { md } { it } {#1} ,
    SlantedFeatures     .code:n = \@@_save_features:nnn { md } { sl } {#1} ,
    BoldFeatures        .code:n = \@@_save_features:nnn { bf } { shape } {#1} ,
    BoldItalicFeatures  .code:n = \@@_save_features:nnn { bf } { it } {#1} ,
    BoldSlantedFeatures .code:n = \@@_save_features:nnn { bf } { sl } {#1} ,
    FontFace            .groups:n = main ,
    UprightFont         .groups:n = main ,
    ItalicFont          .groups:n = main ,
    SlantedFont         .groups:n = main ,
    BoldFont            .groups:n = main ,
    BoldItalicFont      .groups:n = main ,
    BoldSlantedFont     .groups:n = main ,
    UprightFeatures     .groups:n = main ,
    ItalicFeatures      .groups:n = main ,
    SlantedFeatures     .groups:n = main ,
    BoldFeatures        .groups:n = main ,
    BoldItalicFeatures  .groups:n = main ,
    BoldSlantedFeatures .groups:n = main ,
  }
%    \end{macrocode}
%
% \begin{macro}{\@@_semicolon_assign:NNn}
% 将参数 |#3| 中分号前后的内容分别赋值到变量 |#1| 和 |#2| 中。
% 若内容相同,则分号可以省略。若内容为 |-|,则不改变对应变量的当前值。
%    \begin{macrocode}
\cs_new_protected:Npn \@@_semicolon_assign:NNn #1#2#3
  { \@@_assign_auxi:w #3 ; \q_no_value ; \q_stop #1#2 }
\cs_new_protected:Npn \@@_assign_auxi:w #1 ; #2 ; #3 \q_stop
  {
    \use:e
      {
        \@@_assign_auxii:nnNN
          { \tl_trim_spaces:n {#1} }
          { \tl_trim_spaces:n {#2} }
      }
  }
\cs_new_protected:Npn \@@_assign_auxii:nnNN #1#2#3#4
  {
    \str_if_eq:nnF {#1} { - } { \tl_set:Nn #3 {#1} }
    \quark_if_no_value:nTF {#2}
      { \str_if_eq:nnF {#1} { - } { \tl_set_eq:NN #4#3 } }
      { \str_if_eq:nnF {#2} { - } { \tl_set:Nn #4 {#2} } }
  }
\cs_generate_variant:Nn \@@_semicolon_assign:NNn { cc }
%    \end{macrocode}
% \end{macro}
%
% \begin{macro}{\@@_assign:nn}
% 同时保存横排与直排的变量值。
%    \begin{macrocode}
\cs_new_protected:Npn \@@_assign:nn #1
  {
    \@@_semicolon_assign:ccn
      { l_@@_yoko_ #1 _tl }
      { l_@@_tate_ #1 _tl }
  }
\cs_generate_variant:Nn \@@_assign:nn { no }
%    \end{macrocode}
% \end{macro}
%
% \begin{macro}{\@@_assign_font:nn}
% 先将 |*| 号替换掉。
%    \begin{macrocode}
\cs_new_protected:Npn \@@_assign_font:nn #1#2
  {
    \str_if_eq:nnTF {#1} {#2}
      { \@@_assign:nn { font } {#1} }
      {
        \tl_if_blank:nTF {#1}
          { \@@_assign:nn { font } {#2} }
          { \@@_replace_font:nn {#1} {#2} }
      }
  }
\cs_new_protected:Npn \@@_replace_font:nn #1#2
  {
    \tl_set:Nn \l_@@_tmp_tl {#1}
    \tl_replace_all:Nnn \l_@@_tmp_tl { * } {#2}
    \@@_assign:no { font } { \l_@@_tmp_tl }
  }
\cs_generate_variant:Nn \@@_assign_font:nn { o }
%    \end{macrocode}
% \end{macro}

% \begin{macro}{\@@_save_fontface:nnn}
% 保存 \opt{FontFace} 选项的值。\opt{FontFace} 选项的格式与 \pkg{fontspec} 一致。
% \begin{syntax}
%   FontFace = \Arg{series}\Arg{shape}\Arg{file name}
%   FontFace = \Arg{series}\Arg{shape}\{ Font = \meta{file name} , \meta{features} \}
% \end{syntax}
%    \begin{macrocode}
\cs_new_protected:Npn \@@_save_fontface:nnn
  { \@@_save_shape:Nnnn \@@_assign_fontface:nnn }
%    \end{macrocode}
% \end{macro}
%
% \begin{macro}{\@@_save_shape:Nnnn}
% 将选项中设置的字形保存到表中。
%    \begin{macrocode}
\cs_new_protected:Npn \@@_save_shape:Nnnn #1#2#3#4
  { \prop_put:Nnn \l_@@_shape_prop { #2/#3 } { #1 {#2} {#3} {#4} } }
\prop_new:N \l_@@_shape_prop
%    \end{macrocode}
% \end{macro}
%
% \begin{macro}{\@@_save_shape:nnn}
% \begin{variable}[int]{\l_@@_shape_prop}
% 将 \opt{BoldFont} 等选项的值保存到表中。
%    \begin{macrocode}
\cs_new_protected:Npn \@@_save_shape:nnn
  {
    \@@_save_shape_aux:nnn
      { \@@_save_shape:Nnnn \@@_assign_shape:nnn }
  }
%    \end{macrocode}
% \end{variable}
% \end{macro}

% \begin{macro}{\@@_save_features:nnn}
% \begin{variable}[int]{\l_@@_features_prop}
% 将 \opt{BoldFeatures} 等选项的值保存到表中。
%    \begin{macrocode}
\cs_new_protected:Npn \@@_save_features:nnn
  {
    \@@_save_shape_aux:nnn
      { \@@_save_features_aux:nnn }
  }
\cs_new_protected:Npn \@@_save_features_aux:nnn #1#2#3
  { \prop_put:Nnn \l_@@_features_prop { #1/#2 } { {#1} {#2} {#3} } }
\prop_new:N \l_@@_features_prop
%    \end{macrocode}
% \end{variable}
% \end{macro}
%
% \begin{macro}{\@@_save_shape_aux:nnn}
% 辅助函数,用于展开 \tn{bfdefault} 等。
%    \begin{macrocode}
\cs_new_protected:Npn \@@_save_shape_aux:nnn #1#2#3
  {
    \use:e
      {
        \exp_not:n {#1}
          { \use:c { #2 default } }
          { \use:c { #3 default } }
      }
  }
%    \end{macrocode}
% \end{macro}
%
% \begin{macro}[int]{\apfs_set_family:nnn}
% 主要函数。
%    \begin{macrocode}
\cs_new_protected:Npn \apfs_set_family:nnn #1#2#3
  {
    \group_begin:
      \tl_set:Nn \l_@@_family_tl {#1}
      \@@_save_shape:nnn { md } { shape } {#3}
      \tl_if_blank:nF {#2}
        { \apfs_set_filter:nn { fontname } {#2} }
      \@@_declare_family:nn {#2} {#3}
      \@@_declare_shape:n {#3}
    \group_end:
  }
%    \end{macrocode}
% \end{macro}
%
% \begin{macro}{\@@_declare_family:nn}
% 同时声明横排和直排汉字字体族。
%    \begin{macrocode}
\cs_new_protected:Npn \@@_declare_family:nn #1#2
  {
    \@@_check_duplicate:o { \l_@@_family_tl }
    \@@_declare_family_aux:n { yoko }
    \@@_declare_family_aux:n { tate }
  }
\cs_new_protected:Npn \@@_declare_family_aux:n #1
  {
    \use:e
      {
        \apfs_nfss_declare_family:nnn
          { \use:c { l_@@_ #1 _encoding_tl } }
          { \l_@@_family_tl }
          { \exp_not:v { l_@@_ #1 _loading_tl } }
      }
    \tl_clear:c { l_@@_ #1 _loading_tl }
  }
%    \end{macrocode}
% \end{macro}
%
% \begin{macro}{\@@_check_duplicate:n}
%    \begin{macrocode}
\cs_new_protected:Npn \@@_check_duplicate:n #1
  {
%%  TODO
  }
\cs_generate_variant:Nn \@@_check_duplicate:n { o }
%    \end{macrocode}
% \end{macro}
%
% \begin{macro}[int]{\@@_declare_shape:n}
% 设置表中的字形。
%    \begin{macrocode}
\cs_new_protected:Npn \@@_declare_shape:n #1
  {
    \prop_map_inline:Nn \l_@@_shape_prop
      { \@@_declare_shape:nNnnnn { ##1 } ##2 {#1} }
    \prop_if_empty:NF \l_@@_features_prop
      { \@@_declare_shape_feature:n {#1} }
  }
%    \end{macrocode}
% \end{macro}
%
% \begin{macro}[int]{\@@_declare_shape:Nnnnn}
% 实际声明字形。
%    \begin{macrocode}
\cs_new_protected:Npn \@@_declare_shape:nNnnnn #1#2#3#4#5#6
  {
    \group_begin:
      #2 {#1} {#5} {#6}
      \@@_process_size_features:n {#6}
      \@@_declare_shape:nnn { yoko } {#3} {#4}
      \@@_declare_shape:nnn { tate } {#3} {#4}
    \group_end:
  }
%    \end{macrocode}
% \end{macro}
%
% \begin{macro}{\@@_assign_fontface:nnn}
% 实现 \opt{FontFace} 选项。
%    \begin{macrocode}
\cs_new_protected:Npn \@@_assign_fontface:nnn #1#2#3
  {
    \@@_extract_fontname:n {#2}
    \tl_if_empty:NTF \l_@@_fontname_tl
      { \@@_assign_font:nn {#2} {#3} }
      {
        \@@_assign_font:on { \l_@@_fontname_tl } {#3}
        \apfs_set_features:n {#2}
      }
    \prop_remove:Nn \l_@@_features_prop {#1}
  }
\clist_new:N \l_@@_leftover_clist
%    \end{macrocode}
% \end{macro}
%
% \begin{macro}{\@@_assign_shape:nnn}
% 实现 \opt{BoldFont} 等选项。
%    \begin{macrocode}
\cs_new_protected:Npn \@@_assign_shape:nnn #1#2#3
  {
    \@@_assign_font:nn {#2} {#3}
    \prop_pop:NnNT \l_@@_features_prop {#1} \l_@@_tmp_tl
      { \exp_after:wN \@@_assign_shape_aux:nnn \l_@@_tmp_tl }
  }
\cs_new_protected:Npn \@@_assign_shape_aux:nnn #1#2#3
  { \tl_if_blank:nF {#3} { \apfs_set_features:n {#3} } }
%    \end{macrocode}
% \end{macro}
%
% \begin{macro}{\@@_declare_shape_feature:n, \@@_assign_feature:nnn}
% 实现单独的 \opt{BoldFeatures} 等选项。
%    \begin{macrocode}
\cs_new_protected:Npn \@@_declare_shape_feature:n #1
  {
    \prop_map_inline:Nn \l_@@_features_prop
      {
        \@@_declare_shape:nNnnnn { ##1 }
          \@@_assign_feature:nnn ##2 {#1}
      }
  }
\cs_new_protected:Npn \@@_assign_feature:nnn #1#2#3
  {
    \tl_if_blank:nF {#2}
      {
        \@@_assign:nn { font } {#3}
        \apfs_set_features:n {#2}
      }
  }
%    \end{macrocode}
% \end{macro}
%
% \begin{macro}{\@@_process_size_features:n}
% 实现 \opt{SizeFeatures} 选项。
%    \begin{macrocode}
\cs_new_protected:Npn \@@_process_size_features:n #1
  {
    \clist_if_empty:NTF \l_@@_size_features_clist
      { \@@_process_size_features_auxi: }
      { \@@_process_size_features_auxii:n {#1} }
  }
\cs_new_protected:Npn \@@_process_size_features_auxi:
  {
    \tl_gset:Nn \g_@@_yoko_loading_info_tl { <-> }
    \tl_gset:Nn \g_@@_tate_loading_info_tl { <-> }
    \@@_adjust_jfm:
    \@@_construct_fontname:n { yoko }
    \@@_construct_fontname:n { tate }
  }
\cs_new_protected:Npn \@@_process_size_features_auxii:n #1
  {
    \tl_gclear:N \g_@@_yoko_loading_info_tl
    \tl_gclear:N \g_@@_tate_loading_info_tl
    \clist_map_inline:Nn \l_@@_size_features_clist
      { \@@_construct_fontname:nnn ##1 {#1} }
  }
\cs_new_protected:Npn \@@_construct_fontname:nnn #1#2#3
  {
    \group_begin:
      \tl_clear:N \l_@@_fontname_tl
      \apfs_set_size_features:n {#2}
      \tl_if_empty:NF \l_@@_fontname_tl
        { \@@_assign_font:on { \l_@@_fontname_tl } {#3} }
      \@@_adjust_jfm:
      \tl_gput_right:Nn \g_@@_yoko_loading_info_tl { <#1> }
      \tl_gput_right:Nn \g_@@_tate_loading_info_tl { <#1> }
      \@@_construct_fontname:n { yoko }
      \@@_construct_fontname:n { tate }
    \group_end:
  }
\cs_new_protected:Npn \@@_construct_fontname:n #1
  {
    \exp_args:Ncc \@@_process_extension:NN
      { l_@@_ #1 _font_tl }
      { l_@@_ #1 _ext_tl }
    \use:e
      {
        \@@_construct_fontname:Nnnnnnnn
          { \exp_not:c { g_@@_ #1 _loading_info_tl } }
          { \exp_not:v { l_@@_ #1 _scale_tl } }
          { \use:c { l_@@_ #1 _font_tl } }
          { \exp_not:v { l_@@_ #1 _index_tl } }
          { \use:c { l_@@_ #1 _script_tl } }
          { \use:c { l_@@_ #1 _language_tl } }
          { \use:c { l_@@_ #1 _feature_tl } }
          { \use:c { l_@@_ #1 _jfm_tl } }
      }
  }
%    \end{macrocode}
% \end{macro}
%
% \begin{macro}{\@@_adjust_jfm:}
% 如果横排和直排字体的 \opt{JFM} 设置相同,则相应加上 |-h| 和 |-v|。
%    \begin{macrocode}
\cs_new_protected:Npn \@@_adjust_jfm:
  {
    \tl_if_eq:NNT \l_@@_yoko_jfm_tl \l_@@_tate_jfm_tl
      {
        \tl_put_right:Nn \l_@@_yoko_jfm_tl { -h }
        \tl_put_right:Nn \l_@@_tate_jfm_tl { -v }
      }
  }
%    \end{macrocode}
% \end{macro}
%
% \begin{macro}{\@@_construct_fontname:nnnnnnn}
% \ApTeX 的 \tn{jfont} 和 \tn{tfont} 语法如下^^A
% \footnote{\url{https://github.com/clerkma/ptex-ng/blob/master/src/test/test-otf-features.tex}}。
% \begin{verbatim}
% syntax:
%   \jfont\t=name
%   name        = "ot:" file_name file_index? script_lang? gsub_spec? ":" jfm_name
%   file_index  = "[" number "]"
%   script_lang = "(" script? lang? ")"
%   script      = script_tag
%   lang        = "/" lang_tag
%   gsub_spec   = ";" fea_list
%   fea_list    = (fea_tag ",")* [fea_tag | "*"]
%
%   script_tag: https://www.microsoft.com/typography/otspec/scripttags.htm
%   lang_tag:   https://www.microsoft.com/typography/otspec/languagetags.htm
%   fea_tag:    https://www.microsoft.com/typography/otspec/featuretags.htm
%
%   '*' for all gsub featurs
% examples:
%   \jfont\t=ot:yumin.ttf;jp90,hojo:upjisr-h
%   \tfont\t=ot:yumin.ttf;vert:upjisr-v
%   \jfont\t=ot:simsun.ttc[1]:upjisr-h
%   \jfont\t=ot:SourceHanSansTC-Normal.otf:uprml-h
% TODO:
%   GPOS features will be supported in next release.
% \end{verbatim}
%    \begin{macrocode}
\cs_new_protected:Npn \@@_construct_fontname:Nnnnnnnn #1#2#3#4#5#6#7#8
  {
    \tl_gput_right:Nx #1
      {
        \c_@@_prefix_tl
        \tl_if_empty:nF {#2} { s * [ \fp_eval:n {#2} ] }
        #3
        \tl_if_empty:nF {#4} { [ \int_eval:n {#4} ] }
        \tl_if_empty:nF { #5#6 }
          { ( #5 \tl_if_empty:nF {#6} { / #6 } ) }
        \tl_if_empty:nF {#7} { ; #7 }
        : #8
      }
  }
\tl_new:N \l_@@_yoko_loading_info_tl
\tl_new:N \l_@@_tate_loading_info_tl
%    \end{macrocode}
% \end{macro}
%
% \begin{macro}{\@@_process_extension:NN}
% 如果字体名字已经包含扩展名,就不重复加入。
%    \begin{macrocode}
\cs_new_protected:Npn \@@_process_extension:NN #1#2
  { \tl_if_empty:NF #2 { \@@_extension_auxi:NN #1#2 } }
\cs_new_protected:Npn \@@_extension_auxi:NN #1
  {
    \exp_args:Ne \@@_extension_auxi:nNN
      { \exp_args:No \str_lowercase:n {#1} } #1
  }
\cs_new_protected:Npn \@@_extension_auxi:nNN #1
  {
    \clist_map_inline:Nn \l_@@_extension_clist
      {
        \tl_if_in:nnT { #1 $ } { ##1 $ }
          { \clist_map_break:n { \use_none:nnn } }
      }
    \tl_put_right:No
  }
\clist_new:N \l_@@_extension_clist
\clist_set:Nx \l_@@_extension_clist
  { \tl_to_str:n { .ttf , .otf , .ttc } }
%    \end{macrocode}
% \end{macro}
%
% \begin{macro}{\@@_declare_shape:nnn}
% 最后将参数展开后传给 \pkg{NFSS} 的命令。
%    \begin{macrocode}
\cs_new_protected:Npn \@@_declare_shape:nnn #1#2#3
  {
    \use:e
      {
        \apfs_nfss_declare_shape:nnnnnn
          { \use:c { l_@@_ #1 _encoding_tl } }
          { \l_@@_family_tl }
          {#2}
          {#3}
          { \exp_not:v { g_@@_ #1 _loading_info_tl } }
          { \exp_not:v { l_@@_ #1 _loading_tl } }
      }
  }
%    \end{macrocode}
% \end{macro}
%
% \begin{macro}{\@@_extract_alias:nn}
% 先从表中查询,如果查不到,就不改变。
%    \begin{macrocode}
\cs_new_protected:Npn \@@_extract_alias:nn #1#2
  {
    \@@_assign:nn {#1} {#2}
    \@@_extract_alias_aux:ccc
      { l_@@_ #1 _alias_prop }
      { l_@@_yoko_ #1 _tl }
      { l_@@_tate_ #1 _tl }
  }
\cs_new_protected:Npn \@@_extract_alias_aux:NNN #1#2#3
  {
    \tl_if_eq:NNTF #2#3
      {
        \prop_get:NoNT #1 {#2} #2
          { \tl_set_eq:NN #3#2 }
      }
      {
        \prop_get:NoNT #1 {#2} #2 { }
        \prop_get:NoNT #1 {#3} #3 { }
      }
  }
\cs_generate_variant:Nn \@@_extract_alias_aux:NNN { ccc }
%    \end{macrocode}
% \end{macro}
%
% \begin{macro}{\newfontscript}
% \begin{macro}{\apt_def_script_alias:n}
%    \begin{macrocode}
\cs_new_protected:Npn \apt_def_script_alias:n
  { \prop_set_from_keyval:Nn \l_@@_language_alias_prop }
\NewDocumentCommand \newfontscript { }
  { \prop_put:Nnn \l_@@_script_alias_prop }
\prop_new:N \l_@@_script_alias_prop
%    \end{macrocode}
% \end{macro}
% \end{macro}
%
% \begin{macro}{\newfontlanguage}
% \begin{macro}{\apt_def_language_alias:n}
%    \begin{macrocode}
\cs_new_protected:Npn \apt_def_language_alias:n
  { \prop_set_from_keyval:Nn \l_@@_language_alias_prop }
\NewDocumentCommand \newfontlanguage { }
  { \prop_put:Nnn \l_@@_language_alias_prop }
\prop_new:N \l_@@_language_alias_prop
%    \end{macrocode}
% \end{macro}
% \end{macro}
%
%    \begin{macrocode}
\apt_def_script_alias:n
  {
     CJK                   = hani ,
     CJK~Ideographic       = hani ,
     Default               = DFLT ,
     Hangul                = hang ,
     Hangul~Jamo           = jamo ,
     Hiragana~and~Katakana = kana ,
     Hiragana/Katakana     = kana ,
     Kana                  = kana ,
  }
%    \end{macrocode}
%
%    \begin{macrocode}
\apt_def_language_alias:n
  {
    Chinese~Hong~Kong      = ZHH ,
    Chinese~Simplified     = ZHS ,
    Chinese~Traditional    = ZHT ,
    Japanese               = JAN ,
    Korean                 = KOR ,
  }
%    \end{macrocode}
%
% \begin{macro}{\setCJKfamilyfont}
%    \begin{macrocode}
\NewDocumentCommand \setCJKfamilyfont
  { +m +O { } +m +O { } }
  {
    \use:e
      {
        \apfs_set_family:nnn
          {#1}
          { \tl_if_blank:nF {#2} { #2 , } #4 }
          {#3}
      }
  }
%    \end{macrocode}
% \end{macro}
%
%    \begin{macrocode}
\apfs_set:n { JFM = upschrm-h ; upschrm-v }
%    \end{macrocode}
%
%
%    \begin{macrocode}
%</package>
%    \end{macrocode}
%
% \end{implementation}
%
% \Finale
%
\endinput
